\documentclass{ctexart}
\usepackage{ctex}
\usepackage{graphicx}
\usepackage{mathtools}
%\usepackage{datetime}

\graphicspath{{figures/}}

\title{10 Classic Machine Learning Algorithms }
\author{Mr Wu}
\date{\today}

\begin{document}
	\maketitle
	介绍机器学习十大经典算法,并实现。
	\section{Logistic Regression}
	\section{KNN}
	\section{SVM}
	\section{Decision Tree}
	\section{Random Forests}
	\section{Naive Bayes}
	\section{K-means}
	\section{Adaboost}
	\section{Markov}
	\section{EM}
	EM算法:用极大似然的思想来估计模型中的隐变量,然而由于隐变量未知,似然函数无法求出解析解,因此引出EM算法。首先假设一个模型参数,就能求出隐变量的期望,用这个期望作为隐变量的当前估计值。在这个基础上,将似然函数最大化,可得到新的模型参数。根据新的模型参数又可以求出隐变量新的期望,如此循环,直到收敛。
\end{document}