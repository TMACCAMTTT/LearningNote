\documentclass{ctexart}
\usepackage{ctex}
\usepackage{graphicx}
\usepackage{mathtools}

\graphicspath{{figures/}}

\title{MV}
\author{Mr Wu}
\date{\today}

\begin{document}
	\maketitle
	\section{引言}
	对多个不同特征集合的多视角学习快速发展,有良好的理论基础,且取得了巨大的实际成功。
	
	多视图学习涉及由多个不同特征集表示的数据的机器学习问题。这种学习机制的出现很大程度上是被实际应用中数据的属性所驱动的。在实际应用中,样本被不同的特征集合或不同的视角来描述。例如,在多媒体内容理解中,多媒体段可以通过其视频和音频信号同时进行描述。在web页面分类中,web页面可以由文档文本本身描述,也可以由指向此页面的超链接的文本来描述。另一个例子是,在基于内容的web图像检索中,对象是由图像和图像周围文本的视觉特征同时描述的。此外值得注意的一点是,当不存在自然的特征分割时,通过人工制造分割可以提升表现。因此,多视图学习是一个非常有前途的课题,具有广泛的适用性。
	
	CCA和协同训练是早期多视图学习中的两个代表性技巧。目前,多视图学习的概念已经渗透到多个机器学习的分支中,一大批多视图学习算法被提出。多视图学习的应用范围包括降维,半监督学习,监督学习,主动学习,集成学习,迁移学习,聚类等。
	
	本文的目标是回顾多视图学习领域的关键进展,并为今后的研究提供有用的建议。
	\section{多视图学习的理论}
	我们将当前的多视图学习理论分为四类:CCA,协同训练的有效性,协同训练的泛化误差分析,其它多视图学习方法的泛化误差分析。这些理论至少可以部分回答以下三个问题:
	
	1.为什么多视图学习是有用的;
	
	2.基本假设是什么;
	
	3.应该怎样执行多视图学习。
	\subsection{CCA}
	典型相关性分析(CCA)由Hotelling提出,为两视角数据集找出两个线性变换,来最大化变换之后变量的相关性。之后被扩展到多视角的数据。在这里我们只讨论两个视角的数据。
	
	
\end{document}