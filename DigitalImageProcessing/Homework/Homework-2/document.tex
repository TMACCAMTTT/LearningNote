\documentclass{ctexart}
\usepackage{ctex}
\usepackage{graphicx}

\title{Distinctive Image Features From Scale Invariant Keypoints}
\author{Mr Wu}
\date{\today}

\begin{document}
	\maketitle
	\section{摘要}
	本文提出了一种从图像中提取独特不变特征的方法,该方法可用于在对象或场景的不同视角之间执行可靠匹配。这些特征对于图像比例和旋转是不变的,并且能在大范围的仿射失真,3D视角变化,噪声增加,光照变化中提供稳健的匹配。这些特征非常独特,因为单个特征可以高概率地与来自许多图像的大型特征数据库的正确匹配。本文还介绍了利用这些特征进行目标识别的方法。首先使用快速最近邻居算法将各个特征与已知目标的特征数据库相匹配,然后进行霍夫变换以识别属于单个对象的聚类,最后通过最小二乘解决方案对一致姿势参数进行验证来进行识别。这种识别方法可以稳健地识别杂波和遮挡中的对象,同时实现接近实时的性能。
	\section{引言}
	图像匹配是计算机视觉中许多问题的基本方面,包括目标或场景识别,从多个图像求解3D结构,立体对应和运动跟踪。 本文介绍了具有许多属性的图像特征,这些属性使它们适合于匹配目标或场景的不同图像。 这些特征对于图像缩放和旋转是不变的,并且对于照明和3D视角的变化部分不变。它们在空间和频率域都能很好地定位,降低了由于遮挡,杂波或噪声造成的破坏的可能性。 使用有效的算法可以从典型图像中提取大量特征。 此外,这些特征非常独特,允许单个特征与大型特征数据库的高概率正确匹配,为目标和场景识别提供基础。
	
	通过采用级联过滤方法,将提取这些特征的成本降至最低,在这种方法中,更昂贵的操作只应用于通过初始测试的位置。以下是生成图像特征集的主要计算步骤:
	
	1.尺度空间极值检测:计算的第一阶段搜索所有比例和图像位置。 通过使用高斯差分函数来有效地实现它,以识别对于尺度和方向不变的潜在兴趣点。
	
	2.关键点定位:在每个候选位置,都需要一个详细的模型来确定位置和尺度。关键点是根据它们的稳定性来选择的。
	
	3.
\end{document}